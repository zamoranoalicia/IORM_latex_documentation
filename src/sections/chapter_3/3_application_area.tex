%%
%% Author: alicia
%% 13-05-18
%%

% Preamble
\chapter{Definición del Problema y Análisis}
\noindent El capítulo previo, equipó con los conocimientos necesarios sobre dominio del problema.
\noindent Este capítulo discute el problema en sí, exponiendo los aspectos más importantes y define un conjunto
\noindent de requerimientos vitales para encarar el problema.\\
\indent    La Sección 3.1 Provee la definición del problema.
\noindent Posteriormente la Sección 3.2 analiza esta definición. La Sección 3.3 lista los requerimientos
\noindent que son extractados de las secciones anteriores.
\section{Definición del Problema}
\noindent \textit{Escaso contenido semántico en las transformaciones de datos entre el sistema de tipos
                    utilizado en un lenguaje de programación orientado a objetos provoca bajo rendimiento
                    en el desarrollo de un sistema.}

\section{Análisis del Problema}
\noindent Esta sección describe los problemas que necesitan ser resueltos, para solucionar el problema principal.\\
\indent De forma general, los subproblemas que surgen a consecuencia del problema definido en la sección anterior,
\noindent se centran esencialmente en las diferencias que existen entre el \textit{Modelo Orientado a Objetos} y el
\noindent \textit{Modelo Relacional}, dichas diferencias, dificultan la aplicación de características fundamentales
\noindent de uno o ambos modelos, al momento de implementar modelos de datos.\\ Por ejemplo en un lenguaje orientado a objetos,
\noindent un objeto tiene un \textit{identificador}, este es asignado al él al momento de su creación, el desarrollador no
\noindent interviene en la creación y asignación de dicho valor, por otro lado la forma de identificar una tupla en el modelo
\noindent relacional es mediante la definición y creación de una \textit{llave primaria}, para la cual, el desarrollador deberá
\noindent especificar explícitamente el tipo y el dominio.\\ Para la solución de este problema existen patrones de diseño
\noindent de datos que pueden ser aplicados a una solución genérica creando una capa de abstracción adicional, que maneje
\noindent la creación de el \texit{identificador} de un objeto al momento de su instanciación, sin necesidad de requerir
\noindent trabajo adicional por el desarrollador.
\section{Requerimientos}
