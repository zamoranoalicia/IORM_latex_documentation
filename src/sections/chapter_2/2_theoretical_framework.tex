%%
%% Author: Alicia Zamorano
%% 13-05-18
%%

% Preamble
\chapter{Marco teórico}
\noindent Este capítulo provee las pautas más importantes y toda la información requerida para
\noindent comprender esta memoria. Si el lector posee conocimientos previos acerca de Idioms en ER,
\noindent el Modelo Relacional y el Paradigma Orientado a Objetos puede omitir libremente este
\noindent capítulo. \\
\indent  La sección 2.1 de este capítulo explica brevemente que son los Idioms en ER y que
\noident proveen estos a los diseñadores de bases de datos. La sección 2.2 otorga una explicación
\noident rápida del Modelo Relacional y sus características relevantes. Por último la sección 2.3
\noident da a conocer de forma concisa los principales atributos del Paradigma Orientado a Objetos (POO).
\section {Idioms en ER}
\noident Las primeras etapas de la construcción de un sistema de información, más propiamente
\noident la etapa de análisis, usualmente requiere la definición de un modelo de persistencia.
\noident Para esta definición se emplean estructuras preconcebidas, con la ventaja de que estas estructuras
\noident son formaciones sintácticas con semántica correcta, esto permite el reuso de estas estructuras, posibilitando
\noident la formalización de un modelo de alto nivel. \\
\indent  Los Idioms en ER son construcciones semánticas basadas en un conjunto de estructuras sintácticas formalmente
\noident definidas en lenguaje ER las cuales al componerse forman estructuras mucho más complejas.\\
\noident Dado que estas son construcciones semánticas estas no pueden ser descompuestas en símbolos sintácticos,
\noident de igual forma al ser construcciones semánticas estas no pueden ser definidas como soluciones completas,
\noident por tanto, no deben considerarse como patrones de diseño.\\
\noident Es importante denotar que los Idioms poseen atributos tales como:
\begin{itemize}
    \item Ser un conjunto finito de transformaciones. Se tienen siete transformaciones definidas, las cuales serán revisadas en la siguiente sección.
    \item La semántica de un Idiom no altera la semántica original de sus componentes, es decir se preserva su correctitud.
    \item Toda construcción efectuada con Idioms contiene cualidades deseables como correctitud en la navegabilidad y normalización.
\end{itemize} \\
\noident De esta manera estas construcciones añaden propiedades deseables, las cuales las convierten en un recurso eficiente al
\noident momento de abstraer un modelo de datos[Flores, 2006].
\subsection{Clasificación de los Idioms}
\begin{enumerate}
    \item \textit {Clasificación o Catálogo:} La entidad-tipo B clasifica los posibles valores
    que puede tener la entidad-tipo A en los valores de su atributo
    B. En otras palabras, la entidad-tipo B se constituye en un
    catálogo de recursos de entidades que tipifica.
    \item \textit {Composición:} Una entidad B, es el resultado de la composición
    de dos entidades C y A, cuyo dinamismo es el que permite la
    creación de entidades B. En otras palabras se usa este idiom,
    cuando se quiere registrar la historia de las acciones de C y A.
    \item \textit {Reflexión simple:} Entidades de tipo A, se relaciones con sus
    semejantes también de tipo A.
    \item \textit {Reflexión compuesta:} Entidades de tipo E, se relacionan con
    entidades de tipo E con relaciones de la forma.
    \item \textit {Is a:} Una entidad de tipo H es una (is a) entidad
    de tipo I.
    \item \textit {Maestro - detalle:} Una entidad M es detallada por N entidades.
    \item \textit {Básico:} Idiom Básico o Unidad de construcción.
\end{enumerate}
\section {Modelo Relacional}
\noindent El modelo relacional es el resultado del trabajo gestado por Edgar Codd en la década de 1960,
\noindent el cual cimentó su investigación en la \textit {teoría matemática de conjuntos y la lógica de predicados}, enfocado en una construcción
\noindent conocida como relación, logró desarrollar el Modelo Relacional que posteriormente fué introducido en un artículo
\noindent de investigación de IBM el año 1970[Harrington, 2009].\\
\noindent Este modelo describe los datos en términos de \textit{relaciones, atributos o tipos y tuplas}
\noindent en el modelo relacional estos son implementados en un contexto de SQL(\textit{Structured Query Language})
\noindent como tablas, columnas y filas respectivamente[Hay, 2006].\\
\begin{center}
    \begin{table}[H]
        \centering
        \begin{tabular}{|c|c|c|}
            \hline
            Relacional      &             & Contexto SQL\\
            \hline
            Relación        & \rightarrow & Tabla\\
            Atributo o Tipo & \rightarrow & Columna\\
            Tupla           & \rightarrow & Fila\\
            \hline
        \end{tabular}
        \caption{Representación de las características del Modelo Relacional en el Contexto SQL(Elaboración propia)}
    \end{table}
\end{center}
\noindent A continuación se remarca características especificas del contexto SQL que són tomadas en cuenta en el modelo
\noindent relacional.\\
\subsubsection{Columnas}
\begin{itemize}
    \item \textit{Unicidad:} Cada columna debe poseer un nombre único.
    \item \textit{Dominio:} Cada columna esta obligada a pertenecer a un dominio específico, de esta forma se asegura la permisibilidad
          de datos para ese atributo.
    \item \textit {Inexistencia de orden:} No existe el concepto de orden en la definición de columnas, dado que su representación
          esta basada en un \textit{conjunto}, este por definición matemática no toma en cuenta el orden de los elementos.
\end{itemize}
\subsubsection{Filas}
\begin{itemize}
    \item \textit{Valores univaluados}
    \item \textit{Unicidad:} No pueden existir filas duplicadas en una relación
    \item \textit{No existe el concepto de posición}
    \item \textit{Introduce el concepto de Llaves:} Tanto llaves primarias, como llaves foráneas pueden ser
                  parte de una fila, estas podrían formarse como resultado de la combinación de varias columnas,
                  y son indispensables para identificar de forma exclusiva a cada fila.
\end{itemize}
\noindent De la misma manera el modelo relacional sostiene que existen tres componentes importantísimos en el modelo \textit {estructura e integridad}.
\subsection{Estructura}
\noindent El elemento estructural fundamental es, por supuesto, la relación, una relación esta definida sobre un tipo
\noindent (referido asimismo como dominio), el cual es básicamente una abstracción de cualidades similares denominadas atributos,
\noindent los cuales poseen propiedades definidas como:
\begin{itemize}
    \item Unicidad en los nombres de tipos
    \item Homogeneidad en lo valores, es decir los valores de un tipo están obligados a pertenecer a un mismo dominio.
    \item No existencia el concepto de posición, esto resulta en diferentes disposiciones sin ningún impacto en los datos
\end{itemize}
\subsection{Integridad de Datos}
\noindent La integridad de datos es la certeza de correctitud y consistencia de los datos.\\
\noindent Dos restricciones de integridad \textit(integrity constrainsts) se hallan originalmente definidas en el modelo relacional
\noindent formulado por Codd, dichas restricciones incluyen:
\begin{itemize}
    \item \textit {Integridad en las Relaciones:} El primer trabajo realizado por Codd precisaba que los valores en las
                   relaciones no podían ser \textit{nulos}. Posteriormente se introdujo el concepto de \textit{nulo} en los
                   modelos de datos, sin embargo se mantuvo la restricción en la cual un tipo definido como
                   identificador no puede contener un valor nulo.
    \item \textit {Integridad Referencial:} Es otro de los conceptos importantes que introdujo Codd, dado un valor  de un tipo
                   que haga referencia a otra relación, este valor debe existir en dicha relación.
\end{itemize}
\section {Paradigma Orientado a Objetos}
\noindent Conocida como el sucesor de la programación estructurada, la programación orientada a objetos(en adelante citada como POO)
\noindent es un paradigma de programación que tiene como base fundamental el concepto de \textit{objeto}, que básicamente combina datos
\noindent e instrucciones para la creación de este.\\
\noindent Un objeto contiene en si mismo \textit{atributos} y \textit{comportamiento} también conocido como \textit{métodos},
\noindent en la POO los datos y la lógica de programación están combinados, esta combinación puede ocurrir a diferentes niveles de
\noindent \textit{granularidad} permitiendo interacción con una basta cantidad de objetos[Kindler, 2011].\\
\noindent La POO posee cuatro principios fundamentales \textit{Abstracción, Encapsulamiento, Herencia y Polimorfismo},
\noindent sobre los cuales se define un modelo orientado a objetos.[Perry, 2016]\\
\begin{enumerate}
    \item \textit{Abstracción:} Es el proceso de  simplificación de las propiedades de un objeto,
          \noindent desechando así, detalles irrelevantes para la implementación, seleccionando solo los aspectos
          \noindent más esenciales de un objeto.
    \item \textit{Encapsulamiento:} Limita la visibilidad y acceso a los atributos de un objeto dado, de esta forma
          \nonident es posible controlar el uso y la modificación de dichos campos.
    \item \textit{Herencia:} Es la cualidad que tienen los objetos de heredar atributos, posibilitando así la especialización
          \nonindent no aspectos no contemplados en objetos diseñados de forma más generalizada, así mismo evitando la duplicación
          \nonindent de código en caso de existir múltiples especializaciones de un objeto.
    \item \textit{Polimorfismo:} En esencia, polimorfismo denota, que los objetos que pertenecen a la misma rama jerárquica,
           \nonindent pueden manifestar un comportamiento diferente para un mismo requerimiento.
\end{enumerate}
\subsection {Patrones de Diseño}
\noindent El diseño orientado a objetos, es considerado una tarea sustancial al momento de emprender un proyecto de software,
\noindent más aún si se desea elaborar un diseño de alta calidad que incluya \textit{reusabilidad y flexibilidad}, para ello es necesario
\noindent definir los objetos que serán representados en clases, de la misma manera es deseable alcanzar la \textit{granularidad}
\noindent adecuada al momento de crear jerarquías de clase, interfaces, clases abstractas y establecer apropiadamente las relaciones
\noindent que existen entre ellas.\\
\subsubsection {¿Como se describe un patrón de diseño?}
\noindent La forma más acertada para describir un patrón de diseño es a través de \textit {notaciones gráficas},
\noindent no obstante estás no proporcionan una solución integra para un problema dado, es imprescindible, además de
\noindent el registro de las clases y sus relaciones, el registro de los parámetros sobre los cuales se tomo la decisión,
\noindent es decir, que influencias se tiene sobre un patrón de diseño y las alternativas que existen a este.\\
\noindent Para alcanzar este objetivo, se demarco un catálogo de patrones de diseño, clasificados por la forma de encarar
\noindent un problema dado y proveer una solución.\\
\noindent Existen 23 patrones de diseño definidos, cada uno de ellos tiene una variación de acuerdo a la \textit{granularidad}
\noindent y el nivel de abstracción.
\noindent Estos están clasificados de acuerdo a dos criterios, el primero es el propósito, y el segundo
\noindent pero no menos importante es el alcance(Gamma, Helm, Johnson, Vlissides, 1997).\\
\begin{center}
    \begin{table}[H]
        \begin{tabular}{|c|c|c|c|c|}
            \cline{3-5}
            \multicolumn{1}{c}{} & \multicolumn{1}{c|}{} & \multicolumn{3}{c|}{\textbf{Propósito}}\\
            \cline{3-5}
            \multicolumn{1}{c}{} & \multicolumn{1}{c|}{} & \multicolumn{1}{c|}{\textit{Creational Pattern}} &
                                   \multicolumn{1}{c|}{\textit{Structural Pattern}} &
                                   \multicolumn{1}{c|}{\textit{Behavioral Pattern}}\\
            \hline
            \textbf{Alcance} & \textbf{Clase}  & Factory Method   & Adapter      & Interpreter, \newline Template Method\\ \cline{2-5}
                             & \textbf{Objeto} & Abstract Factory & Adapter      & Chain Responsability\\
                             &                 & Builder          & Bridge       & Command \\
                             &                 & Prototype        & Composite    & Iterator\\
                             &                 & Singleton        & Decorator    & Mediator\\
                             &                 &                  & Flightweight & Observer\\
                             &                 &                  & Facede       & Memento\\
                             &                 &                  & Proxy        & State\\
                             &                 &                  &              & Strategy\\
                             &                 &                  &              & Visitor\\
            \hline
        \end{tabular}
        \caption{Clasificación de los Patrones de Diseño[Gamma, Helm, Johnson, Vlissides, 1997]}
    \end{table}
\end{center}
\noindent