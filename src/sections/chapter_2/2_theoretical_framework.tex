%%
%% Author: alicia
%% 13-05-18
%%

% Preamble
\chapter{Marco teórico}
\noindent Este capítulo provee las pautas más importantes y toda la información requerida para
\noindent comprender esta memoria. Si el lector posee conocimientos previos acerca de Idioms en ER,
\noindent el Modelo Relacional y el Paradigma Orientado a Objetos puede omitir libremente este
\noindent capítulo. \\
\indent  La sección 2.1 de este capítulo explica brevemente que son los Idioms en ER y que
\noident proveen estos a los diseñadores de bases de datos. La sección 2.2 otorga una explicación
\noident rápida del Modelo Relacional y sus características relevantes. Por último la sección 2.3
\noident da a conocer de forma concisa los principales atributos del Paradigma Orientado a Objetos (POO).
\section {Idioms en ER}
\noident Las primeras etapas de la construcción de un sistema de información, más propiamente
\noident la etapa de análisis, usualmente requiere la definición de un modelo de persistencia.
\noident Para esta definición se emplean estructuras preconcebidas, con la ventaja de que estas estructuras
\noident son formaciones sintácticas con semántica correcta, esto permite el reuso de estas estructuras, posibilitando
\noident la formalización de un modelo de alto nivel. \\
\indent  Los Idioms en ER son construcciones semánticas basadas en un conjunto de estructuras sintácticas formalmente
\noident definidas en lenguaje ER las cuales al componerse forman estructuras mucho más complejas.\\
\noident Dado que estas son construcciones semánticas estas no pueden ser descompuestas en símbolos sintácticos,
\noident de igual forma al ser construcciones semánticas estas no pueden ser definidas como soluciones completas,
\noident por tanto, no deben considerarse como patrones de diseño.\\
\noident Es importante denotar que los Idioms poseen atributos tales como:

\begin{itemize}
    \item Ser un conjunto finito de transformaciones. Se tienen siete transformaciones definidas, las cuales serán revisadas en la siguiente sección.
    \item La semántica de un Idiom no altera la semántica original de sus componentes, es decir se preserva su correctitud.
    \item Toda construcción efectuada con Idioms contiene cualidades deseables como correctitud en la navegabilidad y normalización.
\end{itemize} \\
\noident De esta manera estas construcciones añaden propiedades deseables, las cuales las convierten en un recurso eficiente al
\noident momento de abstraer un modelo de datos (Flores, 2006).
\subsection{Clasificación de los Idioms}
\begin{enumerate}
    \item \textit {Clasificación o Catálogo:} La entidad-tipo B clasifica los posibles valores
    que puede tener la entidad-tipo A en los valores de su atributo
    B. En otras palabras, la entidad-tipo B se constituye en un
    catálogo de recursos de entidades que tipifica.
    \item \textit {Composición:} Una entidad B, es el resultado de la composición
    de dos entidades C y A, cuyo dinamismo es el que permite la
    creación de entidades B. En otras palabras se usa este idiom,
    cuando se quiere registrar la historia de las acciones de C y A.
    \item \textit {Reflexión simple:} Entidades de tipo A, se relaciones con sus
    semejantes también de tipo A.
    \item \textit {Reflexión compuesta:} Entidades de tipo E, se relacionan con
    entidades de tipo E con relaciones de la forma.
    \item \textit {Is a:} Una entidad de tipo H es una (is a) entidad
    de tipo I.
    \item \textit {Maestro - detalle:} Una entidad M es detallada por N entidades.
    \item \textit {Básico:} Idiom Básico o Unidad de construcción.
\end{enumerate}
\section {Modelo Relacional}
\noindent El modelo relacional es el resultado del trabajo gestado por Edgar Codd en la década de 1960,
\noindent el cual cimentó su investigación en la \textit {teoría matemática de conjuntos y la lógica de predicados}, enfocado en una construcción
\noindent conocida como relación, logró desarrollar el Modelo Relacional que posteriormente fué introducido en un artículo
\noindent de investigación de IBM el año 1970 (Harrington, 2009).\\
\noindent Este modelo describe los datos en términos de \textit{relaciones, atributos y tuplas}
%\noindent estos son referidos en el modelo conceptual como \textit{entidades de clase, atributos y cantidad de ocurrencias};
\noindent en el modelo relacional estos son implementados como tablas, columnas y filas respectivamente(Hay, 2006).\\
\noindent De la misma manera sostiene que existen tres componentes importantísimos en el modelo \textit {estructura, integridad y manipulación}.
\subsection {Estructura}
\noindent El elemento estructural fundamental es, por supuesto, la relación, una relación esta definida sobre un tipo
\noindent (referido asimismo como dominio), el cual es básicamente una abstracción de cualidades similares denominadas atributos,
\noindent los cuales poseen propiedades definidas como:
\begin{itemize}
    \item Unicidad en los nombres de tipos
    \item Homogeneidad en lo valores, es decir los valores de un tipo están obligados a pertenecer a un mismo dominio.
    \item No existencia el concepto de posición, esto resulta en diferentes disposiciones sin ningún impacto en los datos
\end{itemize}
\noindent Una base de datos \textit {relacional} es aquella cuya estructura lógica esta
\noindent compuesta de ``relaciones".
\section {Paradigma Orientado a Objetos}
