%%
%% Author: alicia
%% 13-05-18
%%

% Preamble
\chapter{Introducción}
\noindent Las bases de datos son un aspecto fundamental en el diseño de sistemas, los modelos orientados a objetos,
\noindent se fundan en pensar a cerca de los problemas a resolver empleando modelos que se han organizado tomando como base conceptos del mundo real,
\noindent la transformación del primero al segundo da como resultado una base de datos orientada a objetos virtual, sobre la base de datos relacional , sin embargo esta transformación lleva consigo una perdida importante del contenido semántico de los modelos.
\section{Antecedentes}
\noindent El mapeo objeto relacional es una técnica de programación para convertir datos entre sistemas de tipos, utilizando un lenguaje de programación orientado a objetos y base de datos relacional con la ayuda de un motor de persistencia de objetos [Flores, 2006].
\noindent Existen muchos sistemas que realizan esta transformación como \textbf{Hibernate} utilizado en la programación orientada a objetos en java, \textbf{Doctrine} y \textbf{Propel} que son destacados en el medio,
\noindent los citados anteriormente no toman en cuenta los idioms para su transformación [Fowler, 2010]\\
\section{Definición del Problema}
\noindent La transformación de un modelo relacional a un modelo orientado a objetos proporciona una transformación de una
\noindent base de datos orientada a objetos virtual[Fowler, 2010] \\
\noindent En la programación orientada a objetos, las tareas de gestión de datos son implementadas generalmente por la
\noindent manipulación de objetos, los cuales son casi siempre valores no escalares.
\noindent Muchos productos populares de base de datos, tales como los Sistemas de Gestión de Bases de datos SQL, solamente
\noindent pueden almacenar y manipular valores escalares como integers y strings, organizados en tablas normalizadas.
\noindent Incongruencia entre el modelo relacional y el modelo orientado a objetos. El núcleo del problema reside en traducir
\noindent estos objetos a formas que pueden ser almacenadas en la base de datos para recuperarlas fácilmente, mientras se
\noindent preservan las propiedades de los objetos y sus relaciones[Fowler, 2010]
\noindent Considerando de  suma importancia la calidad de la semántica en las transformaciones se propone el siguiente problema:
\noindent \texbf\textit{Escaso contenido semántico en las transformaciones de datos entre el sistema de tipos utilizado
\noindent en un lenguaje de programación orientado a objetos, provoca bajo rendimiento en el desarrollo de un sistema.}

\section {Objetivos}
\subsection{Objetivo General}
\noindent Definir un modelo para el estudio e implementación de las transformaciones objeto-relacionales,
\noindent proporcionando una transformación de alto nivel, para incrementar el rendimiento en el diseño de persistencia.
\subsection{Objetivos Específicos}
\begin{itemize}
    \item Definir las transformaciones a tráves de E-R Idioms.
    \item Definir funciones genéricas que se puedan aplicar en E-R Idioms.
    \item Crear una arquitectura de transformación modelo-objeto ER-Idiom.
    \item Implementar un prototipo de la transformación para la estructura y para comportamiento
\end{itemize}
\section{Ingeniería del proyecto}
\begin{longtable}{|p{2.8cm}| p{2.8cm}| p{2.8cm}| p{2.8cm}| p{2.8cm}|}
    \hline
    Objetivo General & Causas & Objetivos Especifícos & Actividades & Resultado \\ \hline
    \noindent \multirow{Estudio e implementación de transformaciones objeto relacionales, proponiendo una transformación
    \noindent de alto nivel, para incrementar el rendimiento en el deseño de persistencia.}
    \noindent & {Duplicidad de datos despúes de el mapeo objeto relacional.} & Definir las transformaciones a tráves de E-R Idioms. &
    \parbox
    \begin{itemize}
        \item Realizar un módulo que permita eliminar las tablas duplicadas.
        \item Modelar la BD para los objetos.
    \end{itemize}
    \noindent & Tablas sin datos duplicados después de una transformación objeto relacional.\\ \hline
    \noindent & Inexistencia de funciones genericas aplicables e ER-Idioms} & Definir funciones genericas que se puedan aplicar a ER-Idioms.
    &
    \parbox {2.8cm}{
    \begin{itemize}
        \item Identificar los idioms que tengan características similares a modelos orientados a objetos.
        \item Realizar un modulo que permita la transformación de un objeto.
    \end{itemize}}
    & Mayor cantidad de herramientas con las que se pueda aplicar los idioms.\\ \hline
    &
    {Perdida de contenido semántico después de una transformación objeto relacional.}
    &
    Crear una arquitectura de transformación modelo objeto ER-Idiom. &
    \parbox{2.8cm} {
    \begin{itemize}
        \item Realizar la BD para las transformaciones.
        \item Identificar los idioms que tengan caracteristicas similares a los objetos.
    \end{itemize}}
    &
    Transformaciones de objetos con mayor contenido semántico. \\ \hline
    &
    {Falta de persistencia de los objetos después de una transformación al modelo relacional.} &
    Implementar el prototipo de la transformación para la estructura y para el comportamiento.
    &
    \parbox{2.8cm}{
    \begin{itemize}
        \item Realizar la BD para las transformaciones.
        \item Identificar los idioms que tengan caracteristicas similares a los objetos.
        \item Definir los idioms en términos de estructuras y en términos de funciones.
    \end{itemize}}
    &
    Objetos persistentes después de la transformación al modelo relacional.\\ \hline
\end{longtable}
\section {Innovación Tecnológica}
\noindent Esta investigación tendrá un aporte importante debido a que no existen documentos similares y además es un aporte
\noindent significativo al trabajo llevado a cabo con respecto a los idioms en BD.
\section {Justificación}
\noindent Permitir otros trabajos de investigación, afianzando el concepto de idioms, para así poder establecer una base
\noindent solida a la investigación[Flores, 2006]
\noindent No existen librerías conocidas documentos similares para la aplicación de Idioms.
\noindent A pesar de la existencia de métodods de diseño maduros, patrones de diseño y otras herramientas, estos no reflejan una solución especializada para el manejo de la persistencia y más aún ellos no son una referencia para el diseño ER y relacional[1].
\noindent Este proyecto beneficiará a los desarrolladores, comunidad cientifíca, investigadores de el área de informatica,
\noindent brindando solidez a un concepto que se está manejando.
\section {Alcances del Proyecto.}
\begin{itemize}
    \item El prototipo podrá demostrar la transformación de un Idiom a un modelo orientado a objetos.
    \item El prototipo solo realizará la aplicación para un grupo de idioms determinados.
    \item El proyecto permitirá la definición de Idioms en términos de estructura en terminos de sus definiciones.
    \item El proyecto aplicará sus estructuras sobre datos genéricos.
\end{itemize}
\section{Limites del Proyecto}
\begin{itemize}
    \item Los prototipos son para fines demostrativos.
\end{itemize}